%! Author = slott
%! Date = 11/1/25

% Preamble
\documentclass{beamer}

% To add guillemots for use in UML images.
\usepackage[T1]{fontenc}

% Packages
\usepackage{amsmath}
\usepackage{amsthm}
\usepackage{hyperref}
\usepackage{minted}
\usepackage{tikz}
\usepackage{tikzscale}
\usepackage{qrcode}

% \usetheme{PaloAlto}
% \usetheme{Berkeley}
\usetheme{Goettingen}
\usecolortheme{seagull}

\title{DSL Design with Python}
\subtitle{Using TTRPG Examples}
\author{S.Lott}
\institute{\texttt{\underline{https://fosstodon.org/@slott56}}\linebreak{}\texttt{\underline{https://github.com/slott56}}}
\date{1-Nov-2025}

% Document
\begin{document}

\frame{
    \titlepage
}

\begin{frame}[label=link]
    \frametitle{Link to Slide Deck Repository}
    \qrcode[link]{https://github.com/slott56/dsl-with-python}
    \hfill
    \underline{https://github.com/slott56/dsl-with-python}
\end{frame}


\begin{frame}
\frametitle{Table of Contents}
\tableofcontents
\end{frame}

\section[A Conundrum]{The DSL Conundrum --- You have a problem...}

    \begin{frame}[fragile]
        \frametitle{You have a Problem...}

        \begin{quote}
        Before a small brick building surrounded by forest.
        A stream flows out of the building and down a gully.
        \end{quote}

        \vspace{\baselineskip}
        Descend to find a door with a giant lock and two keys with labels.

        \vspace{\baselineskip}
        Above the door is this quote:

        \begin{quote}
            Some people, when confronted with a problem, think ``I know, I’ll use regular expressions.'' Now they have two problems.
        \end{quote}

        \vspace{\baselineskip}
        The labels:
        \begin{enumerate}
        \item ``RE's are bad; RE $\subseteq$ DSL $\therefore$ $\forall$ DSL's are bad. $\blacksquare$''

        \item ``They always lie.''
        \end{enumerate}

        \note{Which key?}
    \end{frame}

    \begin{frame}
        \transwipe
        \frametitle{The "RE is two problems" fallacy}

        RE syntax is an ugly jumble of infix and postfix operators.

        \vspace{\baselineskip}
        \textbf{Overuse} of regular expressions can be bad.\\
        (The famous Jamie Zawinski quote was about PERL.)\\

        You know: ``When your only tool is a hammer\dots''\\

        \begin{block}{Benefit: a DSL decomposes a problem.}
        \begin{enumerate}
            \item The Domain-Specific Language\dots
            \item Used to solve the problem.
        \end{enumerate}
        \end{block}

        \note{I started making DSL's because I couldn't understand cost accounting.}

    \end{frame}

    \begin{frame}
        \transwipe
        \frametitle{Are there better RE's?}
        \framesubtitle{A digression}

        Consider Al Sweigart's HUMRE.\

        \vspace{\baselineskip}
        It has all the regular expression features.

        In \textbf{Python} syntax.

        \vspace{\baselineskip}
        HUMRE is a DSL for regular expressions in Python syntax.

        \pause \vspace{\baselineskip}
        \textit{And. Bonus. It supports my thesis.}
    \end{frame}

    \begin{frame}
        \frametitle{Occult DSL's}

        You may have designed one of these DSL's.

        \renewcommand{\arraystretch}{1.5} % Default value: 1
        \begin{center}
        \begin{tabular}{ |l|l| }
        \hline
        \textbf{DSL} & \textbf{Syntax} \\
        \hline
        Configuration file  & TOML or INI    \\
        CLI                 & Shell syntax   \\
        RESTful web service & HTTP plus JSON \\
        \hline
        \end{tabular}
        \end{center}

        Does the underlying object model count as a DSL?

        \vspace{\baselineskip}
        \pause\textit{Stick around and see where we wind up.}

    \end{frame}

\section[Design]{DSL Design}

    \begin{frame}
        \transwipe
        \sectionpage
    \end{frame}

    \begin{frame}
        \frametitle{Behind the entrance door is a dark hallway.}

        The hallway ends at a door, covered in images of boxes and arrows.
        A sign on the door says \textbf{Model}.

        \vspace{\baselineskip}
        On your left, you find a door labeled \textbf{Syntax}.

        \note{Which direction? Syntax? Model?}
    \end{frame}

    \begin{frame}
        \frametitle{The \textbf{Syntax} workroom}

        \begin{columns}[T]
        \column{0.33\textwidth}
        \textbf{Door 1}\\
        Two work tables:
        \begin{itemize}
            \item  \textbf{Tokenizer}\\ tools and libraries and modules.
            \item  \textbf{Parser}\\ more tools and libraries.
        \end{itemize}
        Books by \textbf{DABEAZ} everywhere.\\
        \vspace{0.6\baselineskip}
        A pile of parts labeled \textbf{Lark}.

        \column{0.33\textwidth}
        \textbf{Door 2}\\
        A shelf with packages:
        \begin{itemize}
        \item JSON
        \item YAML
        \item TOML
        \item HUML
        \end{itemize}

        \column{0.33\textwidth}
        \textbf{Archway}\\
        Flanked by two friendly-looking snakes.\\
        \vspace{\baselineskip}
        \textit{[Pythons? Roll an} Intellect \textit{check.]}\\
        \vspace{\baselineskip}
        You entered turning left through the \textbf{Syntax} door\dots\\

        \vspace{0.5\baselineskip}
        The \textbf{Model} door is still ahead of you!
        \end{columns}

        \note{Which door? Build it? Use it? Skip it?}
    \end{frame}

    \begin{frame}
        \frametitle{Why the Python door?}

        Avoid writing yet another parser.

        \begin{block}{Two \textit{whattabout} gremlins appear}
            \begin{itemize}
                \item<+-| alert@+> \textit{Whattabout} security? The DSL is Code!

                \vspace{0.66\baselineskip}
                \uncover<+->{
                A chaotic evil 9th level Sorcerer won't waste time hacking  \mintinline{py}{rot13("vzcbeg bf; bf.flfgrz('sbezng p:')")} into DSL-based content.\\

                The \textbf{whole app} is visible, hackable Python.\\

                Consider a side-quest to query the Sphinx.
                }

                \item<+-| alert@+> \textit{Whattabout} the really ugly data model?

                \vspace{0.66\baselineskip}
                \uncover<+-| alert@+>{Wait for case study 2.}
            \end{itemize}
        \end{block}
    \end{frame}

\section[Case Study 1]{Case Study 1 --- TTRPG ``dice expressions''}
    \begin{frame}
        \sectionpage
    \end{frame}

    \begin{frame}[fragile]
        \frametitle{TTRPG ``dice expressions''}

        \begin{example}
        \verb|3d6+2| \hfill ``roll 3 six-sided dice, add 2''
        \end{example}

        \begin{example}
        \verb|4d8| \hfill ``roll 4 eight-sided dice''
        \end{example}

        There's more, but that's for lower levels of the dungeon.

        \vspace{\baselineskip}
        Which path?

        \begin{itemize}
            \item Turn left and write RE's for the syntax?
            \item Define Python classes and objects?
        \end{itemize}

        \uncover<2>{The syntax isn't Pythonic.}

        \note{Which direction? Syntax? Model?}
    \end{frame}

    \begin{frame}
        \frametitle{How to proceed?}

        \begin{block}{Strategy Tip: Adjust the syntax to be Pythonic}

        From this: \(3d6+2\).

        \vspace{\baselineskip}
        To this: \texttt{3 * D6 + 2}
        \end{block}

        \pause\vspace{\baselineskip}
        Python syntax lets you teleport straight to the data model.

        \pause
        \begin{block}{A \textit{whattabout} gremlin appears}
            But \textit{whattabout} the users?

            \vspace{\baselineskip}
            It's a \texttt{*}. \textit{One character.}

            If it helps them solve their problem, they'll embrace it.
        \end{block}

        \note{I'm old. They embraced COBOL.}
    \end{frame}

    \begin{frame}[fragile]
        \frametitle{The \texttt{Die} class}

        \begin{minted}[autogobble,fontsize=\footnotesize]{py}
            class Die:
                def __init__(self, faces: int = 6, n: int = 1) -> None:
                    ...
                def __rmul__(self, other: Any) -> Die:
                    match other:
                        case int():
                            return Die(self.faces, n * other)
                        case _:
                            return NotImplemented
                def roll(self) -> int:
                    ...

            D6 = D(6)
        \end{minted}

        \vspace{\baselineskip}
        This is the data model.
        \pause
        And, \mintinline{py}{3*D6} is valid DSL.\

    \end{frame}


\section[Case Study 2]{Case Study 2 --- \textbf{OpenD6} spells}
    \begin{frame}
        \sectionpage
    \end{frame}

    \begin{frame}
        \frametitle{Context --- legacy data}

        \textbf{OpenD6} rules have an Open Gaming License (OGL), making them open source with attribution.\\
        Perfect for extensions and customizations.

        \vspace{\baselineskip}
        I wanted \textbf{all} the magical spell definitions.

        \vspace{\baselineskip}\uncover<2->{Which are in PDF's.}
        \uncover<3->{Scanned from a printed copy.}

        \vspace{\baselineskip}
        \uncover<4->{How to capture legacy data?}

        \begin{block}<5->{Strategy Tip: Q\&D (Quick \& Dirty) Model}
            \begin{enumerate}
                \item Define a throw-away data structure to capture legacy content.
                \item Then, work out a more useful, semantically complete DSL.\
            \end{enumerate}
        \end{block}
    \end{frame}

    \begin{frame}
        \frametitle{Legacy Data Model}

        \begin{center}
        \includegraphics[height=0.75\textheight]{legacy_capture.png}
        \end{center}

    \end{frame}

    \begin{frame}
        \frametitle{Expanded DSL Data Model}

        \includegraphics[width=1.0\textwidth]{dsl_model.png}

        \uncover<2->{Yes, it's complicated.}

        \uncover<3->{A (second) bulk conversion was required.}

        \uncover<4->{I omitted some stuff, and it was still vast.}
    \end{frame}


    \begin{frame}[fragile]
        \frametitle{Example DSL statement}
        \begin{minted}[autogobble,fontsize=\scriptsize]{python}
        Spell(
            name="Push",
            skill="Apportation",
            notes="Even the most subtle of things,,,",
            effect=TimeEffect("sends a small object into the future", "10 min"),
            duration=DurationAspect(measure="10 minutes"),
            range=RangeAspect(measure="touch"),
            casting_time=CastingTimeAspect(measure="1 round"),
            speed=SpeedAspect.based_on(("range",), ""),
            other_aspects={
                "gestures": GesturesAspect("Wave one hand ...", "simple"),
                "incantations": IncantationsAspect(
                   "Where did it go?", "sentence"
                ),
            },
        )
        \end{minted}
        \vspace{\baselineskip}
        It has a lot of boilerplate. The API design needs work.
    \end{frame}

    \begin{frame}
        \frametitle{Typical DSL use cases}

        \begin{itemize}
            \item Represent legacy content and rules in the DSL.\

            \item Compute derived values from DSL.\

            (Compare with legacy sources as acceptance test case.)

            \item Present content in RST format for publication.

            \item \alert<2>{Identify special cases and possible errors.}
        \end{itemize}

        \begin{block}<3->{DSL and Learning}
            \begin{itemize}
                \item <4-| alert@+>
                    Unless you're already the leading expert in the problem domain,
                    expect the DSL evolution.

                \item <5-| alert@+>
                    DSL evolution $\equiv$ Learning

                    The SOLID \textbf{Open/Closed Principle} is your friend.
            \end{itemize}
        \end{block}

    \end{frame}

    \begin{frame}
        \frametitle{But \textit{whattabout} the ugly data model?}

        \begin{block}{Strategy Tip: Replace tears with tiers}
            \begin{description}
                \item<+-| alert@+>[Foundation] Your ugly domain model that really captures \textbf{everything} the end users are prattling on about.
                \item<+-| alert@+>[API] A layer on top of the ugly foundation that creates the classes and objects for a pleasant, easier-to-use DSL.\
            \end{description}
        \end{block}

        \begin{block}<+->{This is Pythonic OO design.}
            Design patterns to use:
            \begin{itemize}
                \item Façade [koan 5]: Flat is better than nested
                \item Adapter [koan 8]: Special cases aren't special enough to break the rules
            \end{itemize}
            Want more ideas?
            Run \mintinline{py}{import this} for all the koans.
        \end{block}

    \end{frame}

\section{Conclusion}
    \begin{frame}
        \sectionpage
    \end{frame}

    \begin{frame}
        \frametitle{Tips for DSL design}
        \begin{itemize}
            \item<+-> Use Python as the syntax for a DSL.\
            \item<+-> Capture legacy rules with a Q\&D model.
            \item<+-> As you learn, build a complete DSL model.
            \item<+-> Tears $\rightarrow$ tiers --- layer the model for usability.
        \end{itemize}
        \vspace{\baselineskip}
        \begin{center}
        \begin{alert}{Pursue Pythonic Practices}
        \end{alert}
        \end{center}
    \end{frame}

    \againframe{link}

\appendix
\section{\appendixname}
\begin{frame}
\tableofcontents
\end{frame}

\subsection{Malicious Actors}

    \begin{frame}
        \frametitle{How to handle malicious actors}

        \begin{block}{Are \textbf{all} elements present?}
            \begin{itemize}
                \item Untrusted sources of DSL,
                \item And large DSL docs,
                \item And difficulty in doing code validation.
            \end{itemize}
        \end{block}

        Two strategies:
        \begin{enumerate}
            \item Use JSON or TOML.\ \\
            Use a \texttt{@classmethod} to build DSL model objects from \texttt{dict[str, Any]} content.

            \item Don't allow DSL modules to use \mintinline{py}{import}.\\
            Use the \textbf{Compile-Check-Exec} pattern.
        \end{enumerate}
    \end{frame}

    \begin{frame}[fragile]
        \frametitle{Avoid \texttt{import} with Compile-Check-Exec}
        \begin{block}{Compile}
        \begin{minted}[autogobble,fontsize=\scriptsize]{python}
    source = Path(module).with_suffix(".py").read_text()
    spell_book_code = ast.parse(source, module, "exec")
        \end{minted}
        \end{block}
        \begin{block}{Check}
        \begin{minted}[autogobble,fontsize=\scriptsize]{python}
    visitor = AllImports(source)
    visitor.visit(spell_book_code)
    if visitor.imports:
        raise ValueError(f"import detected: {visitor.imports}")
        \end{minted}
        \end{block}
        \begin{block}{Exec}
        \begin{minted}[autogobble,fontsize=\scriptsize]{python}
    global_defs: dict[str, Any] = {}
    local_vars: dict[str, Any] = {}
    exec("from magic2 import *", global_defs, local_vars)
    exec(source, global_defs, local_vars)
    return local_vars["spells"]
        \end{minted}
        \end{block}
    \end{frame}

    \begin{frame}[fragile]
        \frametitle{The \texttt{AllImports} visitor}
        \begin{minted}[autogobble,fontsize=\scriptsize]{python}
class AllImports(ast.NodeVisitor):
    def __init__(self, source: str) -> None:
        self.source = source
        self.imports: list[str | None] = []

    def visit_Import(self, node: ast.Import) -> None:
        self.imports.append(ast.get_source_segment(self.source, node))

    def visit_ImportFrom(self, node: ast.ImportFrom) -> None:
        self.imports.append(ast.get_source_segment(self.source, node))
        \end{minted}
    \end{frame}

\end{document}
