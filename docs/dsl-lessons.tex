%! Author = slott
%! Date = 28-sep-2025

% Preamble
\documentclass{beamer}

% Packages
\usepackage{amsmath}
\usepackage{tikz}
\usepackage{tikzscale}
\usepackage{minted}
\usepackage{hyperref}
\usepackage{qrcode}
\usepackage[shortcuts]{extdash}

\usetheme{PaloAlto}
\usecolortheme{seagull}
\usemintedstyle{bw}

\title{DSL Lessons}
\subtitle{from Table-Top Role-Playing Games (TTRPG)}
\author{S.Lott}
\institute{\texttt{\underline{https://fosstodon.org/@slott56}}\linebreak{}\texttt{\underline{https://github.com/slott56}}}
\date{28-Sep-2025}

% Document
\begin{document}

\frame{
    \titlepage
}

%\begin{frame}
%    \frametitle{Link to Slide Deck Repository}
%    \qrcode[link]{https://github.com/slott56/writing-tools}
%    \hfill
%    \underline{https://github.com/slott56/writing-tools}
%\end{frame}

\begin{frame}
\frametitle{Table of Contents}
\tableofcontents
\end{frame}

\section{What's this about?}

\begin{frame}
    \frametitle{TTRPG? DSL?}
    \framesubtitle{Wait, what?}

    \begin{itemize}

        \item A Domain-Specific Language (DSL) can be \textbf{very} hard to design.

        \item A Table-Top Role-Playing Game (TTRPG) is a complicated problem domain.
        (And fun. And low-risk.)

        \item Designing a DSL for a TTRPG can provide insight into DSL design.
    \end{itemize}

    We'll look at two examples.
\end{frame}

\section{Some background}

\begin{frame}
    \frametitle{About Me}

    This starts almost half a century ago.

    In the previous millennium.

    \begin{itemize}
        \item First TTRPG: D\&D Original Edition in 1976.
        \item First personal computer: Apple ][+ in 1982.
        \item Started using Python in 1998.
    \end{itemize}
    \vspace{1em}
    TL;DR: I'm an old.

    [I'll try to minimize rambling about the olden days.]
\end{frame}

\begin{frame}
    \frametitle{About DSLs}

    \textbf{Domain-Specific Language}

    Anything \textbf{not} a general-purpose programming language.

    We are surrounded by DSL's.
    \begin{itemize}
        \item HTML
        \item CSS
        \item Makefiles
        \item JSON, YAML, TOML, HUML
        \item Regular Expressions
    \end{itemize}
    \vspace{1em}
    I could go on.

    The number of DSL's is an actual problem.

\end{frame}


\begin{frame}
    \frametitle{Important Consideration}

    \begin{block}<1->{Question to ask yourself}
    Do you need a unique DSL?
    \end{block}

    \begin{block}<2->{Choices}
        \begin{itemize}
            \item Can you get by with TOML/HUML/YAML or JSON?
            \item Is Python all you need?
        \end{itemize}
    \end{block}

\end{frame}

\begin{frame}
    \frametitle{Big Lesson Up Front}

    \begin{alertblock}{Important}
    Python may be all you need.

    \vspace{1em}
    Start there.
    \end{alertblock}

\end{frame}

\section{Starting example}
\begin{frame}
    \frametitle{Dice-Rolling Language}
    \framesubtitle{3d6+2}

    This is an easy place to start.

    \vspace{1em}
    \fbox{\texttt{3d6 + 2}} is a syntax for rolling polydedral dice.
    In this case 3 6-sided die; then adding 2.

    \vspace{1em}
    D\&D used platonic solids: d4, d6, d8, d12, d20.

    Other games added a d10 (which isn't a platonic solid)
\end{frame}

\begin{frame}
    \frametitle{And the complication}
    \framesubtitle{there's always a complication}

    The ``keep'' and ``drop'' dice mechanics.
    Roll 4 dice, keep the top 3. Or drop the lowest 1.

    \fbox{\texttt{4d6k3}} or \fbox{\texttt{4d6d1}}

    \vspace{1em}
    Four cases? Or only two?
    \begin{itemize}
        \item \texttt{k}\textit{n} --- Keep highest \textit{n}
        \item \texttt{d}\textit{n} --- Drop highest \textit{n}
        \item Keep lowest? How do we distinguish keep highest from keep lowest? \(k\wedge\) and \(k\vee\)?
        \item Drop lowest?
    \end{itemize} \pause

    \vspace{1em}
    \begin{alertblock}{Stop. Breathe.}
    Do we need all of this?
    \end{alertblock}

\end{frame}

\begin{frame}[fragile]
    \frametitle{Or. Maybe just Python}
    \begin{minted}[gobble=8]{python}
        weapon_damage = 3 * D6 + 2

        characteristic = (4 * D6).keep(3)
        character = [
            characteristic.roll() for _ in range(6)
        ]
    \end{minted}
    \vspace{1em}
    Looks good to me. \texttt{3*D6+2} vs. \texttt{3d6+2}

    One extra \texttt{*}.
\end{frame}

\begin{frame}[fragile]
    \frametitle{How does that work?}
    \begin{minted}[gobble=8]{python}
        class Die:
            def __init__(self, d: int) -> None:
                ...
            def __rmul__(self, n: int) -> "Die":
                ...
            def __add__(self, n: int) -> "Die":
                ...
            def keep(self, k: int) -> "Die":
                ...
        D6 = Die(6)
    \end{minted}
    The implementation isn't \textit{too} complicated.
\end{frame}

\begin{frame}[fragile]
    \frametitle{Python?}
    \begin{block}{Advantages}
        \begin{itemize}
            \item No parser needed.
            \item Unit tests can be docstrings.
        \end{itemize}
    \end{block}
    \begin{block}{Disadvantages}
        \begin{itemize}
            \item Some extra punctuation.
            \item Maybe use \texttt{kh()} instead of \texttt{keep()}
            Or use the \texttt{\_\_rshift\_\_()} operator:
            \texttt{4 * D6 >> 3}.
        \end{itemize}
    \end{block}
\end{frame}
\begin{frame}
    \frametitle{What have we learned?}
    \framesubtitle{Python's OK.}

    \begin{itemize}
        \item Python object creation syntax is acceptable for declarative syntax.
        \item Python class definitions provide working code behind the object declarations.
        \item You don't have to write a parser.
        \item You can focus on the game rules.
    \end{itemize} \pause

    \vspace{1em}
    \textit{Read: Enterprise Business Rules.}
\end{frame}

\begin{frame}
    \frametitle{Obligatory Evil Super-Genius\textsuperscript{\small{\texttrademark}} Remark}
    Yes. The Evil Super-Genius (ESG\textsuperscript{\small{\texttrademark}})
    can --- and will --- hack your Python-based DSL.

    \vspace{1em}
    They will inject state-changing statements into otherwise declarative code.

    \vspace{1em}
    Look at Sphinx with it's \texttt{conf.py} configuration file.
    People are always hacking it to change the \texttt{sys.path} value.

    \begin{alertblock}<2->{Sarcasm}
    I have mention the remark is sardonic.

    There's always someone who's sure Python\-/as\-/declarative\-/language can't work in practice.

    Even when it does.
    \end{alertblock}

\end{frame}

\section{Bigger example}

\begin{frame}
    \frametitle{The OpenD6 System}

    Why OpenD6?

    The Open Game License (OGL) makes the rules open-source.

    It was new (to me).

    It seems to have a bunch of TTRPG advantages over other games. \pause

    \vspace{1em}
    [Skipping a lot of blah-blah-blah. Cut to the DSL!]
\end{frame}

\begin{frame}
    \frametitle{OpenD6 Magic}
    The OpenD6 Fantasy rules define a bunch of magical spells.

    \vspace{1em}
    There are pages of rules.

    \vspace{1em}
    The gamemaster is given a little bit of advice on the complicated rules for spell definitions.

    \begin{itemize}
        \item An ``Adjusting and Readjusting'' admonition on page 86.
        \item A ``Spell Design'' sheet on page 96.
        \item ``A calculator might also help.'' \textit{Really.}
    \end{itemize}

\end{frame}

\begin{frame}
    \frametitle{DSL Use Cases}

    There are two things I need to do.

    \begin{itemize}
        \item Parse spell definitions from a variety of rule sources.
        \item Create new spells.
    \end{itemize}

    \vspace{1em}
    Ingesting is really important.

    I want the \textit{OpenD6 Fantasy} spells, and the \textit{OpenD6 Magic Guidebook} spells.

    \vspace{1em}
    I'll be adapting some campaign and world-building from the Hero Game System to OpenD6. Why not?
\end{frame}

\begin{frame}
    \frametitle{Strategy}
    \begin{block}{Phase I}
        \begin{enumerate}
            \item Design some V1 Python classes for the underlying model.

                \texttt{Spell}, \texttt{Effect}, \texttt{Attribute}, etc.

            \item Parse some source text and build Python objects.

                Confirm your understanding of the data model.

            \item Make changes as new weirdnesses surface.

                Until you can parse \texttt{everything}.
            \item Add minimal unit tests for the model and the spells.
        \end{enumerate}
    \end{block}
    \vspace{1em}
    You now have some working code that \textit{seems} to fit the available data.
\end{frame}

\begin{frame}[fragile]
    \frametitle{Strategy}
    \begin{block}{Phase II}
        \begin{enumerate}
            \item Design the V2 Python classes.

                Incorporate the lessons learned in parsing.

            \item Write a v1 \(\rightarrow\) v2 translator.

                Uncover numerous subtle issues that were glossed over in v1.
            \item Serialize the final, complete v2 spells as Python objects.
            \item Profit.
        \end{enumerate}
    \end{block}
    The second generation DSL fits all the data.
\end{frame}

\begin{frame}
    \frametitle{Hey! Wait!}
    ``I'm not stupid,'' you say. ``I don't need to write (and discard) a draft DSL to understand the problem domain.''

    \vspace{1em}
    That may be true.

    I wouldn't bet on it.

    \vspace{1em}
    I would bet on writing V1 to learn how the problem domain \textbf{really} works.
\end{frame}

\section{Example}

\begin{frame}[fragile]
    \begin{minted}[gobble=4,fontsize=\scriptsize]{python}
    Spell(
        name="Push",
        skill="Apportation",
        notes="Even the most subtle of things,,,",
        effect=TimeEffect("sends a small object into the future", "10 min"),
        duration=DurationAspect(measure="10 minutes"),
        range=RangeAspect(measure="touch"),
        casting_time=CastingTimeAspect(measure="1 round"),
        speed=SpeedAspect.based_on(("range",), ""),
        other_aspects={
            "gestures": GesturesAspect("Wave one hand ...", "simple"),
            "incantations": IncantationsAspect(
               "Where did it go?", "sentence"
            ),
        },
        other_conditions=[
            ArcaneKnowledgeAspect(0, "time"),
            GenericAspect(difficulty=0, description="Difficulty: 10"),
        ],
    )
    \end{minted}
\end{frame}

\begin{frame}[fragile]
    \frametitle{That looks acceptable}
    Most of the spell details look like this.

    \vspace{1em}
    \begin{minted}[gobble=4,fontsize=\normalsize]{python}
    Spell(
        ...
        duration=DurationAspect(measure="10 minutes"),
        ...
    ),
    \end{minted}

    \vspace{1em}
    The essential \textit{label} = \textit{value} that we expect in TOML, JSON, etc.

\end{frame}

\section{Conclusion}
    \begin{frame}
        \frametitle{Summary}
        Don't invent a DSL from scratch.

        \vspace{1em}
        Start with Python object definitions as your declarative language.

        \vspace{1em}
        Don't write a parser unless there's no alternative.
    \end{frame}
\end{document}
